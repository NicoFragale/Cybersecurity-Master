\section{Web Assembly}

\subsection{Introduction}

WASM is not a programming language, but a binary format generated from other language like C, C++ or Rust. 
WASM permit this hig level language to run efficiently and properly.
It is executed in safe place like browser or other runtime enviroment.

\textbf{It is safe because it runs in isolated sandbox.}

It is used to increase performance in web application as: 
\begin{enumerate}
    \item 3D games in Browser; 
    \item Figma etc\dots
    \item Editing image/video software online\dots
    \item Ai, ML, blockchain, criptoghraphy\dots
    \item Allow to execute C, C++, Rust online;
    \item It can be used on server.
\end{enumerate}

\subsection{Key characteristics}

\begin{enumerate}
    \item Stack-Based (push and pop) \texttt{<->} Does not use registers; \texttt{->} Operations;
    \item Executabel in web broser \texttt{->} using WebAssembly JavaScript API. \texttt{->} API is the only way to communicate from sandbox to outside;
    \item Secure \texttt{->} Sandbox and permission denied to access system resources;
    \item Platform-independent \texttt{->} runs on any device that has WASM runtime.
\end{enumerate}

\begin{lstlisting}[language=wat]
    (func $calcola (param $x i32) (result i32)
        local.get $x
        local.get $x
        i32.mul
        i32.const 2
        i32.mul
        i32.const 1
        i32.add
    )
    \end{lstlisting}

Analyze the example: 
\begin{enumerate}
    \item \textit{func \$calcola (param \$x i32) (result i32) }:
        \begin{enumerate}
            \item \textit{func} it is the key word declaring the function;
            \item \textit{\$calcola} function's name;
            \item \textit{param} it is the key word declaring the parameter;
            \item \textit{\$x} parameter's name;
            \item \textit{i32} indicates the data-type (32 bit integer);
            \item \textit{result i32} indicates the result will be a i32 data type.
            \item \textbf{if \$ is omitted the code will still work.}
        \end{enumerate}
    
    \item \textit{local.get \$x} push X in stack with index 0 (Func Starts wih stack empty);
    \item \textit{local.get \$x} push X in stack with index 1;
    \item \textit{i32.mul} pop 0 and 1 mul, then mul them (both x) and push temporary result in index 0; 
    \item \textit{i32.const 2} push in stack the value 2 as type i32 and index 1;
    \item \textit{i32.mul} pop 0 and 1, then mul them and push as temporary result as index 0;
    \item \textit{i32.const 1} add 1 as i32 in index 1 ;
    \item \textit{i32.add} pop 0 and 1, add index 0 and 1, result is pushed in index 0.
\end{enumerate}

\subsection{Data-Type}

\begin{enumerate}
    \item \textbf{i32} integer with or without sign in 32 bit \texttt{->} (from 0 to 4.294.967.295) or (from -2.147.483.648 to 2.147.483.647);
    \item \textbf{i64} integer with or without sign in 64 bit;
    \item \textbf{f32} floating poin in 32 bit;
    \item \textbf{f64} floating point in 64 bit.
\end{enumerate}

\subsection{Storing Values}

\begin{enumerate}
    \item Stack \texttt{->} push and pop (for operations) of the parameter and costant;
    \item Function context \texttt{->} variabale declared inside the function \texttt{->} Example: \textit{local \$temp i32}; \texttt{->}
    \begin{lstlisting}[language=wat]
        (func $quadrato (param $x i32) (result i32)
        (local $temp i32)  
        local.get $x       
        local.get $x       
        i32.mul            
        local.set $temp    
        local.get $temp    
    )
\end{lstlisting}
    \item Single global memory \texttt{->} Linear memory to handle complex data structure \texttt{->} Used by many functions to store data in long term or to share data beetween more functions \texttt{->} Example:

    \begin{lstlisting}[language=wat]
        (global $contatore (mut i32) (i32.const 0))  
  
        (func $incrementa (result i32)
            global.get $contatore   
            i32.const 1
            i32.add                 
            global.set $contatore   
            global.get $contatore   
        )
        \end{lstlisting}
    \textbf{(mut i32)} mutable variabale type i32 (otherwise immutable during the execution), \textbf{(i32.const 0)} initialized at 0.

\end{enumerate}


